%        File: Resumen.tex
%     Created: Sat Sep 02 11:00 PM 2023 -0
% Last Change: Sat Sep 02 11:00 PM 2023 -0
%
\documentclass[a4paper]{article}
\begin{document}

\section{Modelos de simulacion de sistemas discretos}
\subsection{Simulacion}
Es una imitacion simplificada de un sistema que tiene por objetivo entender
y mejorar el sistema. Entre las simulaciones pueden distinguirse
las estaticas, que son invariantes en el tiempo y las dinamicas, que imitan
al sistema y su comportamiento segun el avance del tiempo.

\subsection{Sistema} %TODO completar definicion
Es un conjunto de partes organizadas para alcanzar un proposito.
Pueden distinguirse las siguientes clases de sistema:
\begin{itemize}
    \item Naturales: Tienen origen en la naturaleza. Ejemplo: Atomo.
    \item Diseñados y fisicos: Son diseñados por el hombre. Ejemplo: auto.
    \item Diseñados y abstractos: Sistemas abstractos diseñados por el
        hombre. Ejemplo: matematicas.
    \item De actividad humana: Un sistema que tiene origen en la actividad
        humana pero no fue diseñado. Ejemplo: familia, ciudad.
\end{itemize}

\subsection{Proposito de una simulacion}
El proposito de hacer una simulacion es obtener un mejor entendimiento e
identificar posibles mejoras para un sistema, esta informacion luego puede
ser utilizada para tomar mejores decisiones sobre el sistema real.

\subsection{La naturaleza de los sistemas de operacion}
Las simulaciones consideradas se usan principalmente para describir
sistemas fisicos diseñados y para sistemas de actividad humana.
Existen muchos sistemas que no pertencen exclusivamente a una de estas
categorias sino que incluyen partes de ambas. Se denomina de forma general
a estos sistemas como sistemas operacionales. Un sistema operacional es
una configuracion de recursos combinados para proveer un bien o servicio.

Estos sistemas pueden estar afectados por la \textbf{variaabilidad}. Esta 
variabilidad puede ser predecible, por ejemplo el cambio de la cantidad
de operadores en un callcenter como respuesta a diferentes volumenes de 
llamadas durante un dia o impredecible, como la tasa de ingresos a una
sala de emergencias de un hospital.

Otra de sus caracteristicas es la \textbf{interconexion}. Los componentes
se afectan unos a otros. Considerando tambien la variabilidad presente
en el sistema, la prediccion del efecto de la interconexion es dificil
de llevar a cabo.

Por otro lado, los sistemas operacionales pueden tener cierta complejidad.
La complejidad puede ser combinatorial, es decir que al aumentar la
cantidad de partes del sistema, la candidad de conexiones entre ellas
aumenta rapidamente.
El otro tipo de complejidad es la dinamica. Esta complejidad surge de
la interaccion de componentes del sistema a traves del paso del tiempo,
lo que produce que una accion tenga efectos muy diferentes en el corto y
largo plazo, que una accion tenga diferentes consecuencias en las diferentes
partes del sistema y que las acciones produzcan efectos contraintuitivos.

La complejidad hace que sea muy dificil predecir el desempeño del sistema
al llevar a cabo acciones.

Es por estas caracteristicas de los sistemas operacionales que puede ser
util emplear simulaciones, ya que estas tienen la capacidad de representar
la variabilidad, interconexion y complejidad.

%TODO
Notas Modelos de simulacion de sistemas discretos y continuos: ocnceptos y metodologias.
Modelo de diseño:
depende de la plataforma y del paradigma elegidos para la simulacion

Simulacion basada en eventos discretos:
el sistema se representa como entidades que fluyen entre actividades, con un itempo entre 
cada una.
el modelo evoluciona en el tiempo y las variables de estado cambian instantaneamente en 
puntos separados de tiempo.
en un sistema continuo las vars de estado cambian en cada par de puntos, en discretos solo en
determinados puntos, en ambos casos el cambio es instantaneo.
Evento: ocurrencia que sucede dentro del alcance del sistema, instantanea, que puede 
cambiar el estado del sistema

Instantaneo: se ordena el cambio de una variable y ese cambio se puede observar en el sistema
sin el paso del tiempo de por medio.

Proceso: secuencia de eventos ordenada por el paso del tiempo. Siempre se definen para una
misma entidad
Actividad: conjunto de operaciones que cambian el estado de una entidad. No pueden ser 
interrumpidas una vez que una entidad las inicia

Los estados de las entidades permanecen constantes entre eventos

Mecanismos de avance de tiempo:
Soluciones a computar el paso del tiempo. Orientado a intervalo o a evento

A intervalo: se divide el tiempo en intervalos regulares y al final de cada uno se evalua
si un evento ocurrio en su transcurso. Si ocurrio se ejecuta el evento. Usado para
simulaciones continuas

A evento: el tiempo de simulacion se avanza hasta alcanzar el proximo evento. Este es el 
evento mas inminente. Usado para simulaciones discretas

Una de las categorias de los mecanismos orientados a eventos es la programacion de eventos:

Componentes y organizacion:
estado del sistema
reloj de simulacion
lista de eventos
contadores estadisticos
rutina de inicializacion
rutina de tienmpo
etc

Estado: coleccion de variables de estados necesarias para describir al sistema en un instante

Reloj de sim:
variable que mantiene el valor actual de tiempo simulado.

Lista de eventos:
contiene el proximo tiempo en el cual ocurrira cada posible tipo de evento del sistema

Contadores est:
variables ocn la informacion estadistica relacionada al comportamiento del sistema

R de inicializacion
inicializa el modelo en tiempo cero: en su estado inicial. Asigna los valores iniciales de
estado y a los contadores est.

R de tiempo:
vincula lista de eventos con el reloj de sim. Determina el evento mas inminente y actualiza
el reloj con el timpo correspondiente a ese evento

Rutina de evento:
actualiza el estado cuando ocurre determinado evento. Debe haber una por cada tipo de evento

Lib de rutinas:
rutinas que generan observaciones aleatorias usando las distribuciones de probabilidad 
asociadas al modelo

Generador de reportes:
calcula estimaciones de medidas de performance usando los contadores estadisticos. se usa
al finalizar la simulacion

Prog principal:
Invoca a las demas rutinas: rutina de tiempo -> r de evento \ldots -> generador de reportes

Requieren una organizacion:
Existen relaciones entre los componentes basadas en flujos de control

El reloj de sim, contadores est, lista de eventos, <falta uno> son alcanzables por todos los 
componentes 

Entidad:
Objeto de interes caracterizado por los valores de sus atributos. Estos atributos son las
propiedades que permiten describir su estado como parte del sistema. Son las propiedades
que interesan medir de la entidad segun los objetivos del modelo.
El estado de una entidad son los valores de sus atributos en un determinado instante.

Los estados de las entidades mas importantes de un sistema definen el estado de este.

Ejercicio: modelo de diseño
Segun el modelo conceptual, cada componente 'incluido' sera una entidad
Por cada entidad se definen atributos, actividades y eventos. Debe haber un evento de inicio
y uno de fin
Debe usarse el 'nivel de detalle' para definir las propiedades
Las propiedades pueden ser marcas temporales, ej: atributo-inicio = t=3

Lineamiento:
prop de conjunto en modelo conceptual -> se usa para determinar propiedades individuales
ej: tiempo entre arribos -> marca de tiempo en que arribo

por cada actividad se define un evento de inicio y de fin
Entre actividades no ocurren eventos

El evento de inicio de una actividad siempre es el de inicio de otra

Una actividad puede tener tiempo 0

Luego se construyen las rutinas de evento
Solo se construyen para eventos incondicionales

Evento condicional:
Dependen del estado de otra entidad para ejecutarse

Evento incondicional:
Ocurrencia aleatoria

Programacion de eventos:
Se hace una descripcion completa de los cambios de estado debidos a eventos incondicional

Los eventos condicionales de forma indirecta se programan dentro de las rutinas de los 
eventos incondicionales

Tambien se identifican eventos equivalentes, esto es, mismo evento en distintas entidades
(mismo comportamiento en el sistema)


\end{document}


