%        File: Resumen.tex
%     Created: Sat Sep 02 11:00 PM 2023 -0
% Last Change: Sat Sep 02 11:00 PM 2023 -0
%
\documentclass[a4paper]{article}
\begin{document}

\section{Modelos de simulacion de sistemas discretos}
\subsection{Simulacion}
Es una imitacion simplificada de un sistema que tiene por objetivo entender
y mejorar el sistema. Entre las simulaciones pueden distinguirse
las estaticas, que son invariantes en el tiempo y las dinamicas, que imitan
al sistema y su comportamiento segun el avance del tiempo.

\subsection{Sistema} %TODO completar definicion
Es un conjunto de partes organizadas para alcanzar un proposito.
Pueden distinguirse las siguientes clases de sistema:
\begin{itemize}
    \item Naturales: Tienen origen en la naturaleza. Ejemplo: Atomo.
    \item Diseñados y fisicos: Son diseñados por el hombre. Ejemplo: auto.
    \item Diseñados y abstractos: Sistemas abstractos diseñados por el
        hombre. Ejemplo: matematicas.
    \item De actividad humana: Un sistema que tiene origen en la actividad
        humana pero no fue diseñado. Ejemplo: familia, ciudad.
\end{itemize}

\subsection{Proposito de una simulacion}
El proposito de hacer una simulacion es obtener un mejor entendimiento e
identificar posibles mejoras para un sistema, esta informacion luego puede
ser utilizada para tomar mejores decisiones sobre el sistema real.

\subsection{La naturaleza de los sistemas de operacion}
Las simulaciones consideradas se usan principalmente para describir
sistemas fisicos diseñados y para sistemas de actividad humana.
Existen muchos sistemas que no pertencen exclusivamente a una de estas
categorias sino que incluyen partes de ambas. Se denomina de forma general
a estos sistemas como sistemas operacionales. Un sistema operacional es
una configuracion de recursos combinados para proveer un bien o servicio.

Estos sistemas pueden estar afectados por la \textbf{variaabilidad}. Esta 
variabilidad puede ser predecible, por ejemplo el cambio de la cantidad
de operadores en un callcenter como respuesta a diferentes volumenes de 
llamadas durante un dia o impredecible, como la tasa de ingresos a una
sala de emergencias de un hospital.

Otra de sus caracteristicas es la \textbf{interconexion}. Los componentes
se afectan unos a otros. Considerando tambien la variabilidad presente
en el sistema, la prediccion del efecto de la interconexion es dificil
de llevar a cabo.

Por otro lado, los sistemas operacionales pueden tener cierta complejidad.
La complejidad puede ser combinatorial, es decir que al aumentar la
cantidad de partes del sistema, la candidad de conexiones entre ellas
aumenta rapidamente.
El otro tipo de complejidad es la dinamica. Esta complejidad surge de
la interaccion de componentes del sistema a traves del paso del tiempo,
lo que produce que una accion tenga efectos muy diferentes en el corto y
largo plazo, que una accion tenga diferentes consecuencias en las diferentes
partes del sistema y que las acciones produzcan efectos contraintuitivos.

La complejidad hace que sea muy dificil predecir el desempeño del sistema
al llevar a cabo acciones.

Es por estas caracteristicas de los sistemas operacionales que puede ser
util emplear simulaciones, ya que estas tienen la capacidad de representar
la variabilidad, interconexion y complejidad.


%TODO: agregar notas de clase
\section*{Recopilacion y analisis de datos}
Se pueden distinguir tres tipos de requerimientos de datos. El primer tipo son 
los datos preliminares o de contexto, estos se emplean para desarrollar un 
entendimiento del sistema a modelar y se veran reflejados luego en el modelo 
conceptual.
El segundo tipo es el de datos necesarios para construir el modelo computacional.
En esta etapa se requieren conjuntos mas grandes de datos mas detallados que 
permitan definir las distribuciones de tiempo y otros patrones en el modelo.
Por ultimo se requieren datos para validar el modelo generado.
Esta informacion puede obtenerse de comparar al modelo con el sistema real,
si este existe.

\subsection*{Obtencion}
Dependiendo de la disponibilidad de los datos se definen tres categorias.
La primera abarca los datos que ya estan disponibles o que ya se conocen, 
pueden provenir de una recopilacion anterior, de dispositivos que registren
datos de forma automatica u otras fuentes. Es necesario controlar que 
los datos tengan el formato adecuado y sean correctos antes de emplearlos.

La categoria B incluye los datos que no se conocen pero pueden recopilarse.

Por ultimo en la categoria C se encuentran los datos que no estan disponibles 
para su recopilacion, que no pueden obtenerse.
Para esta categoria de datos puede trabajarse:
\subsubsection*{Estimar los datos}
\begin{itemize}
    \item Utilizando sistemas similares
    \item Datos estandarizados
    \item Consultando expertos en el area 
    \item Proponiendo una aproximacion propia del equipo de simulacion (adivinar)
\end{itemize}
Estimar los datos añade incertidumbre a la simulacion por lo que deben aclararse
las estimaciones que se usaron y debe hacerse un estudio de sensibilidad, debe 
medirse el efecto de los datos estimados sobre los resultados. Si los efectos son
importantes, no sera util usar una estimacion.

\subsubsection*{Tomarlos como factores experimentales}
Deben ser datos de los que se tenga el control para modificarlos en el
sistema real
Si no es posible resolver el problema de los datos de categoria C de estas formas,
puede revisarse el modelo conceptual o modificarse los objetivos de la simulacion.
De esta forma pueden dejar de requerirse los datos que no se pueden obtener.

\subsubsection*{Datos de categoria A y B}
Si se tienen datos disponibles o si se desea recopilar datos, es fundamental que
se analicen las fuentes de informacion, en el caso de categoria A, debe considerarse
como fueron obtenidos los datos y para que fin. En ambas categorias, ademas, debe
verificarse que la informacion se encuentra en un formato adecuado a la simulacion.

\subsubsection*{Trazas}
Son conjuntos de datos que describen una secuencia de eventos y los tiempos en que 
sucedieron. Usualmente se recopilan del sistema real y luego la simulacion lee 
los eventos de esta secuencia para recrearlos. Son utiles para validar el modelo 
ya que se usan datos historicos, reales.
Dependen de que exista el sistema real y que esten presentes los datos. Puede ser muy
costoso emplear trazas si existen muchas fuentes de variabilidad.

\subsubsection*{Distribuciones empiricas}
Contienen valores o rangos de valores y las frecuencias con las que ocurre cada valor o rango.
Se pueden obtener de las trazas.
Por lo general es conveniente utilizar tiempos entre eventos en vez de cantidades de eventos por
unidad de tiempo ya que en la simulacion se desea recrear cada evento.

\subsubsection*{Distribuciones estadisticas}
Son distribuciones definidas por funciones matematicas. Pueden ser discretas o continuas.
Si se emplean estas distribuciones pueden llevarse a cabo analisis de sensibilidad 
facilmente. Tienen la desventaja de poseer colas muy largas, es decir de poder generar
valores extremos muy grandes ocasionalmente y en algunos casos valores negativos.
No son transparentes para los clientes y no ayudan a generar confiabilidad en el modelo.

\subsubsection*{Seleccion de una distribucion estadistica}
Puede hacerse esta seleccion en base a las propiedades del proceso que se esta modelando.
Esto permite ahorrarse la recoleccion de datos.
Por otro lado, puede ajustarse una distribucion a los datos que se tienen, esto conlleva 
seleccionar una distribucion candidata, determinar los valores de sus parametros y analizar
la 'bondad' del ajuste, es decir, si la distribucion propuesta se parece a los datos recopilados.

Para seleccionar una candidata, se utiliza el histograma de frecuencias de los datos y se lo 
inspecciona para determinar a que distribucion se parece, ademas se deben considerar las propiedades 
del proceso.

Los parametros se calculan en base a los datos del histograma. % Agregar mas informacion de chi-cuadrado, K-S y A-D p 355
% Buscar paginas de wikipedia de esto pq no hay un pingo en los libros
La \textbf{bondad de ajuste} puede probarse graficamente, por ejemplo construyendo un histograma basado en la
distribucion propuesta y comparandolo al original o haciendo una comparacion quantil-quantil entre ambos
conjuntos de datos.

Por otro lado, puede hacerse la prueba por metodos analiticos como chi-cuadrado, Kolmogorov-Smirnov o Anderson-Darling.
\subsubsection*{Chi-cuadrado}
Formaliza la idea de comparar el histograma de la muestra con el de la distribucion a la que se quiere ajustar.
Requiere una muestra de gran tamaño y agrupar los valores en clases de equivalencia, estas clases se crean de forma 
arbitraria y depende cuales se designen, el resultado de la prueba puede variar. Ademas, es necesario agrupar clases
con muy poca frecuencia hasta llegar a un minimo.

\subsubsection*{Kolmogorov-Smirnov}
Formaliza la idea de hacer un analisis grafico quantil-quantil.
Es util en casos de muestras reducidas y de no disponerse de parametros calculados sobre la muestra.

\subsubsection*{Anderson-Darling}
Similar a K-S ya que se comparan las frecuencias acumuladas de la muestra y de la distribucion para llevar a cabo
la prueba. Por otro lado, este test utiliza otro metodo para medir las diferencias entre ambas, diferente a tomar
la maxima diferencia. Esto permite hacer a este test mas sensible y preciso, principalmente en las colas de las 
distribuciones.


\section{Generacion de numeros aleatorios} 
Los numeros aleatorios deben ser uniformes e independientes. Deben ser muestras independientes una
de la otra de una distribucion uniforme entre 0 y 1.
Estas dos caracteristicas se pueden probar, la uniformidad mediante pruebas de bondad que comparen una muestra o
secuencia con una distribucion uniforme y la independencia comparando la correlacion entre valores de la muestra y 
la correlacion esperada que es 0.

\subsection*{Metodos de generacion de numeros pseudo-aleatorios}
Los metodos de generacion de numeros aleatorios deben poseer las siguientes caracteristicas:
\begin{itemize}
    \item El metodo debe ser rapido
    \item Portable a diferentes plataformas
    \item Debe tener un ciclo largo, haciendo referencia a la cantidad de numeros generados antes de repetir una 
    secuencia
    \item Debe aproximar las caracteristicas de los numeros aleatorios, incluyendo parametros como la varianza
\end{itemize}

\subsubsection*{Congruencia lineal} % agregar los otros metodos
Permite determinar el ciclo maximo y la velocidad de generacion al elegir los valores de a, c y m.

\section{Generacion de variables aleatorias}
Se utilizan numeros aleatorios para muestrear valores que pertenezcan a una determinada distribucion de
probabilidad de interes.

\subsection*{Metodo de la transformada inversa}
Consiste en hallar la funcion de probabilidad acumulada de la distribucion $F(X)$, luego igualarla a $R$,
un numero aleatorio que va a ser generado y luego despejar $X$ para obtener una funcion inversa $F^{-1}(R) = X$.
Con esta funcion se pueden obtener valores de la variable X a partir de los numeros R que sigan la distribucion 
empleada. 
Este metodo se utiliza cuando es posible y sencillo despejar la funcion inversa. Tambien se emplea para distribuciones
empiricas.
En el caso de distribuciones discretas se emplean los valores de la tabla de frecuencias acumuladas para determinar 
los extremos del subintervalo de $[0,1]$ que corresponde a cada valor discreto. Luego, dependiendo de a que subintervalo
pertenezca cada numero aleatorio, se calcula el valor de la variable.

\subsection*{Metodo del rechazo}
Para cada valor de variable aleatoria que se requiera, se genera un numero aleatorio. Se compara este numero a una condicion
y si la satisface, el valor se acepta como valor de la variable, por otro lado, si no cumple la condicion, se lo rechaza y
se genera un nuevo numero aleatorio para comenzar el proceso nuevamente. Dependiendo de la distribucion se empleara una 
condicion diferente o se empleara mas de un numero aleatorio por iteracion. Ademas, el numero aleatorio debe ser utilizado 
en alguna funcion que de como resultado la variable.
Este metodo permite generar variables de una distribucion de Poisson.

\subsection*{Metodos de propiedades especiales (Distribucion Normal)}
Existen metodos que no son generales sino que son caracteristicos de cada distribucion y se emplean para obtener valores 
de variables con distribuciones mas complejas o que son dificiles de obtener con los demas metodos.
Este es el caso de la distribucion normal. % Agregar mas sobre este metodo particular (no esta en Banks) Buscar en las diapositivas o algun apunte/resumen

\section*{Diseño de experimentos}
Se plantean interrogantes y se diseñan experimentos para determinar que pruebas es necesario hacer y como llevarlas a cabo para responder
a los interrogantes empleando la menor cantidad de recursos y tiempo posibles.

\subsection*{Experimento}
Es un cambio que se hace sobre las condiciones o las entradas de un sistema o proceso con el fin de medir el efecto del cambio sobre 
sus propiedades o variables de respuesta.

\subsection*{Unidad experimental}
Es la pieza o muestra sobre la que se realiza el experimento, debe ser representativa del sistema o proceso.

\subsection*{Variable de respuesta}
Son propiedades o caracteristicas que permiten conocer los efectos de los experimentos a partir de su medicion.

\subsection*{Factor}
Es una variable o caracteristica que se puede controlar o cambiar y por lo tanto son las variables que se emplean 
en los experimentos. Para elegir que factores emplear se debe recurrir al conocimiento que se tenga sobre el 
sistema en estudio, es decir, los factores que se estime que puedan influir en las respuestas.

\subsection*{Nivel}
Es un valor determinado que se asigna a un factor dentro de un experimento. Si el experimento involucra varios factores,
a una combinacion de niveles se le llama \textbf{tratamiento}.

\subsection*{Error aleatorio}
Es la parte de la variabilidad que se observa en las variables de respuesta que no se debe a los cambios de los 
experimentos sino que son errores naturales del proceso en estudio, son aleatorios. Pueden provenir de factores
que no fueron tenidos en cuenta por considerarse de poca importancia.

\subsection*{Error experimental}
Es el error que comete el experimentador al llevar a cabo los experimentos, si es importante el error no se podra
determinar el efecto de los factores en el sistema.

\subsection*{Diseño}
Es el numero de experimentos, sus tratamientos y la cantidad de repeticiones por cada uno.

\subsection*{Interacciones}
Dependiendo del diseño de experimento que se emplee, pueden existir interacciones entre los factores que tengan 
un efecto sobre las variables de respuesta ademas de los efectos individuales de los factores.

\subsection*{Planeacion}
En primer lugar se define el objeto de estudio, un problema si lo hay y se hace una investigacion preliminar para
conocer el dominio.
Luego se eligen las variables de respuesta que mejor caractericen al problema y que se puedan medir de forma
precisa y consistente.
Se determinan los factores y sus niveles

\subsection*{Analisis}
Ya que los experimentos se hacen sobre muestras y no la poblacion, debe hacerse un analisis estadistico para
determinar si los efectos de los factores son lo suficientemente grandes para garantizar que tambien se dan los 
efectos en la poblacion. Para esto se emplean los ANOVAs.

\subsection*{Interpretacion}
Se contrastan las conjeturas o conocimientos preliminares sobre el problema con los resultados de los experimentos.

\subsection*{Conclusion}
Se proponen medidas a implementar para solucionar el problema o mejorar el funcionamiento del sistema.

\subsection*{Principios del diseño de experimentos}
Si no se siguen estos principios, la validez de los datos que se obtengan de los experimentos sera perjudicada.

\subsubsection*{Aleatorizacion}
Las diferentes corridas de experimentos y la seleccion de material para las pruebas deben hacerse de forma 
aleatoria con el fin de que los errores causados por factores ambientales o temporales se distribuyan de forma 
aleatoria en todas las pruebas y se aumente la independencia de las pruebas.

\subsubsection*{Repeticion}
Llevar a cabo cada experimento mas de una vez, es decir, ejecutar varias veces cada tratamiento de forma aleatoria 
contribuye a la distribucion de los errores de forma aleatoria y ademas permite medir y distinguir entre la variabilidad 
debida a los efectos de factores y la variabilidad de los errores aleatorios o experimentales.

\subsection*{Diseño unifactorial}
Se estudia el efecto de un unico factor y se considera que este y el error aleatorio son las unicas dos fuentes de
variabilidad.
Se consideran ademas 4 hipotesis:
\begin{itemize}
    \item La suma de todos los efectos del factor es 0
    \item La suma de todos los errores es 0
    \item Los errores son independientes y uniformemente distribuidos
    \item Los errores tienen una distribucion normal
\end{itemize}
Como primer analisis puede hacerse una asignacion de la variacion que consiste en calcular que porcentaje de 
la variacion total ocupa la variacion del factor, teniendo en cuenta que el resto del porcentaje lo ocupa 
la variacion del error.

\subsection*{ANOVA}
Es el analisis de varianza, es un analisis estadistico que permite determinar si el efecto de un factor es 
relevante o no comparandolo con el efecto del error aleatorio.
La hipotesis nula es que las medias de cada nivel del factor son iguales, la hipotesis alternativa es que 
al menos un par de medias son diferentes, en este ultimo caso, se considera que el factor tiene un efecto con 
importancia estadistica sobre la variable de respuesta. Esto significa que los cambios en el nivel del factor 
generan cambios significativos en la respuesta.
El metodo emplea sumas de cuadrados de los efectos (SSA, SSE), sus cuadrados medios (MSA, MSE) y se utiliza una 
variable de Fisher para relacionar las varianzas. Esta variable es la que se utiliza para determinar si se 
cumple la hipotesis nula, comparandola con una tabla.

\subsection*{Intervalos de confianza}
Se utilizan intervalos de confianza como indicadores del grado de variabilidad de los efectos.
Se calculan con el cuadrado medio del error y la distribucion de Student. Si un intervalo 
contiene al valor cero, es probable que el efecto del nivel no sea relevante.
Tanto el ANOVA como los intervalos de confianza pueden ser calculados solo en diseños con
repeticiones ya que si no se hacen repeticiones no puede calcularse el error.

\subsection*{Diseño factorial completo}
Se emplea para conocer el efecto de mas de un factor y las interacciones entre ellos.
Tiene las mismas hipotesis que el diseño unifactorial y ademas:
\begin{itemize}
    \item La suma de los efectos de la interaccion de A y B en los niveles de A es 0
    \item La suma de los efectos de la interaccion de A y B en los niveles de B es 0
\end{itemize}

\subsection*{Diseño factorial $2^k$}
Consiste en un diseño con varios factores pero con solo dos niveles por factor. 
Se suele utilizar al comienzo e los experimentos o en situaciones donde realizar un 
diseño factorial completo puede ser muy costoso o complejo. Puede utilizarse para descartar 
factores que no tengan efectos sobre la variable de respuesta y luego considerar mas niveles 
en un diseño factorial completo.

\subsection*{Diseño factorial fraccionario $2^{k-p}$}
Se utiliza para estudiar el efecto en situaciones donde la cantidad de factores es demasiado
grande. 'P' es la candidad o fraccion de combinaciones de factores que se dejan sin estudiar 
en el diseño.
Las combinaciones que no se estudian son las que estan superpuestas o confundidas con otras 
combinaciones (sus efectos).
Utilizar este diseño implica dividir las $2^k$ combinaciones en $2^p$ particiones y ejecutar 
las corridas de simulacion en una de ellas para relacionar los resultados con las demas combinaciones.
La desventaja de crear muchas particiones es que se pierde mas informacion, por lo tanto debe 
tomarse una decision considerando el costo del diseño, la cantidad de factores y cuanta informacion 
es aceptable perder para simplificar el diseño.

\end{document}


