%        File: Resumen.tex
%     Created: Sat Sep 02 11:00 PM 2023 -0
% Last Change: Sat Sep 02 11:00 PM 2023 -0
%
\documentclass[a4paper]{article}
\begin{document}

\section{Modelos de simulacion de sistemas discretos}
\subsection{Simulacion}
Es una imitacion simplificada de un sistema que tiene por objetivo entender
y mejorar el sistema. Entre las simulaciones pueden distinguirse
las estaticas, que son invariantes en el tiempo y las dinamicas, que imitan
al sistema y su comportamiento segun el avance del tiempo.

\subsection{Sistema} %TODO completar definicion
Es un conjunto de partes organizadas para alcanzar un proposito.
Pueden distinguirse las siguientes clases de sistema:
\begin{itemize}
    \item Naturales: Tienen origen en la naturaleza. Ejemplo: Atomo.
    \item Diseñados y fisicos: Son diseñados por el hombre. Ejemplo: auto.
    \item Diseñados y abstractos: Sistemas abstractos diseñados por el
        hombre. Ejemplo: matematicas.
    \item De actividad humana: Un sistema que tiene origen en la actividad
        humana pero no fue diseñado. Ejemplo: familia, ciudad.
\end{itemize}

\subsection{Proposito de una simulacion}
El proposito de hacer una simulacion es obtener un mejor entindimiento e
identificar posibles mejoras para un sistema, esta informacion luego puede
ser utilizada para tomar mejores decisiones sobre el sistema real.

\subsection{La naturaleza de los sistemas de operacion}
Las simulaciones consideradas se usan principalmente para describir
sistemas fisicos diseñados y para sistemas de actividad humana.
Existen muchos sistemas que no pertencen exclusivamente a una de estas
categorias sino que incluyen partes de ambas. Se denomina de forma general
a estos sistemas como sistemas operacionales. Un sistema operacional es
una configuracion de recursos combinados para proveer un bien o servicio.

Estos sistemas pueden estar afectados por la \textbf{variaabilidad}. Esta 
variabilidad puede ser predecible, por ejemplo el cambio de la cantidad
de operadores en un callcenter como respuesta a diferentes volumenes de 
llamadas durante un dia o impredecible, como la tasa de ingresos a una
sala de emergencias de un hospital.

Otra de sus caracteristicas es la \textbf{interconexion}. Los componentes
se afectan unos a otros. Considerando tambien la variabilidad presente
en el sistema, la prediccion del efecto de la interconexion es dificil
de llevar a cabo.

Por otro lado, los sistemas operacionales pueden tener cierta complejidad.
La complejidad puede ser combinatorial, es decir que al aumentar la
cantidad de partes del sistema, la candidad de conexiones entre ellas
aumenta rapidamente.
El otro tipo de complejidad es la dinamica. Esta complejidad surge de
la interaccion de componentes del sistema a traves del paso del tiempo,
lo que produce que una accion tenga efectos muy diferentes en el corto y
largo plazo, que una accion tenga diferentes consecuencias en las diferentes
partes del sistema y que las acciones produzcan efectos contraintuitivos.

La complejidad hace que sea muy dificil predecir el desempeño del sistema
al llevar a cabo acciones.

Es por estas caracteristicas de los sistemas operacionales que puede ser
util emplear simulaciones, ya que estas tienen la capacidad de representar
la variabilidad, interconexion y complejidad.
\end{document}


